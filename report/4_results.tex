\section{Results}
The following sections will compare the different implementations of the cache
coherence protocols. The quantitative analysis section will compare the MESI
protocol and the Dragon protocol while the advanced section will evaluate the
benefits of the optimization of the MESI protocol.

\subsection{Quantitative Analysis}
% Benchmarks:

% Numbers are [Cache size, associativity, block_size] All 3 DS: 4KiB, 2, 32 with
% MESI and Dragon Fastest benchmark with MESI and Dragon:
% - 4KiB, 1, 32
% - 4KiB, Full, 32
% - 1KiB, 2, 32
% - 8KiB, 2, 32
% - 4KiB, 2, 16
% - 4KiB, 2, 64

The analysis in this section is based on the default configuration described in
the project description. The MESI protocol and the Dragon protocol have been evaluated
on every benchmark trace while varying one of either the cache size, the block
size or the associativity. Each plot contains two subplots, one that shows the
absolute values for both the MESI protocol and the Dragon protocol (the bottom
subplot) and one that shows the relative difference of the absolute values of
the two protocols (the top subplot), because they are quite similar. This is useful to
distinguish the overall pattern when varying a parameter as well as see which of
the protocols performed best. The difference is always calculated using
\[
    P^\text{MESI}_i - P^\text{Dragon}_i
\]
where $ P^\text{<protocol>}_i $ is the value for each protocol, for each
parameter value $i$ in each plot. A negative difference means that the Dragon protocol
required more cycles, a positive means that MESI was slower than Dragon.
Since it takes quite some time to run each
benchmark trace, there will only be two shifts for each parameter. This already
constitutes a large number of tests,
\[
    \underbrace{2}_\text{2 protocols} \times \underbrace{3}_\text{3 traces} \times \underbrace{2}_\text{2 shifts} \times \underbrace{3}_\text{3 parameters} + \underbrace{3 \times 2}_\text{default} = 42
\]
without accounting for the advanced tests. Figure~\ref{fig:cache_size} shows
the performance for each protocol when varying the cache size.


\begin{figure}[H]
    \centering
    \begin{subfigure}[b]{0.33\textwidth}
        \centering
        \incfig{cache_size_blackscholes}
        \caption{Blackscholes.}\label{fig:cache_size_blackscholes}
    \end{subfigure}%
    \hfill
    \begin{subfigure}[b]{0.33\textwidth}
        \centering
        Cache size\par\medskip
        \incfig{cache_size_bodytrack}
        \caption{Bodytrack.}\label{fig:cache_size_bodytrack}
    \end{subfigure}%
    \hfill
    \begin{subfigure}[b]{0.33\textwidth}
        \centering
        \incfig{cache_size_fluidanimate}
        \caption{Fluidanimate.}\label{fig:cache_size_fluidanimate}
    \end{subfigure}
    \hfill
    \caption{The graphs show the number of executed cycles when varying the size of the cache. There are three different settings for the cache size, 1024 bytes, 4096 bytes and 8192 bytes. The associativity is 2 and the block size is 32 bytes.}\label{fig:cache_size}
\end{figure}


We can see that the MESI protocol performs better for the Blackscholes benchmark,
while the Dragon protocol performs better for the Bodytrack and the Fluidanimate
benchmark when the cache size is 1024. However, the difference is only significant
for this cache size, and as we increase the cache size, both protocols have more
or less the same performance. The overall pattern is that the bigger the cache size
is, the fewer clock cycles are required, while the impact of increasing the cache
size seems bigger in the Blackscholes. Bigger cache sizes lead to fewer cache misses
which will reduce the number of clock cycles. However, there is a tradeoff, since
bigger memories have higher access time. The result of the simulator is not affected
by this tradeoff.

In Figure~\ref{fig:associativity} we can again see that the overall pattern of
increasing in this case the associativity decreases the number of clock cycles
for each benchmark trace. The impact is again greater in the Blackscholes benchmark.
However, the Dragon protocol performs better in the Blackscholes benchmark, and the
MESI protocol performs better in the Fluidanimate benchmark. Higher degree of associativity
will in principle reduce the number of cache misses. However, there is again a tradeoff
since a higher degree of associativity will require more parallel searches for the correct
cache line. The result of the simulator is again not affected by this tradeoff.

In Figure~\ref{fig:block_size}, the overall pattern is not the same in every subplot.
In the Blackscholes benchmark it seems like increasing the block size increases the
number of clock cycles. For the Bodytrack benchmark and  the Fluidanimate benchmark,
increasing the block size decreases the number of clock cycles, indicating that these
two benchmarks deal with a lot of consecutive data where it is beneficial to load bigger
chunks of data. The MESI protocol performs better in the Blackscholes benchmark and the
Fluidanimate benchmark, but the differences are more subtle the higher the block size is.

\begin{figure}[H]
    \centering
    \begin{subfigure}[b]{0.33\textwidth}
        \centering
        \incfig{associativity_blackscholes}
        \caption{Blackscholes.}\label{fig:associativity_blackscholes}
    \end{subfigure}%
    \hfill
    \begin{subfigure}[b]{0.33\textwidth}
        \centering
        Associativity\par\medskip
        \incfig{associativity_bodytrack}
        \caption{Bodytrack.}\label{fig:associativity_bodytrack}
    \end{subfigure}%
    \hfill
    \begin{subfigure}[b]{0.33\textwidth}
        \centering
        \incfig{associativity_fluidanimate}
        \caption{Fluidanimate.}\label{fig:associativity_fluidanimate}
    \end{subfigure}
    \hfill
    \caption{The graphs show the number of executed cycles when varying the associativity. There are three different settings for the associativity, 1 (direct mapped), 2 (2-set-associative) and 128 (fully associative). The block size is 32 bytes and the cache size is 4096 bytes.}\label{fig:associativity}
\end{figure}

\begin{figure}[H]
    \centering
    \begin{subfigure}[b]{0.33\textwidth}
        \centering
        \incfig{block_size_blackscholes}
        \caption{Blackscholes.}\label{fig:block_size_blackscholes}
    \end{subfigure}%
    \hfill
    \begin{subfigure}[b]{0.33\textwidth}
        \centering
        Block size\par\medskip
        \incfig{block_size_bodytrack}
        \caption{Bodytrack.}\label{fig:block_size_bodytrack}
    \end{subfigure}%
    \hfill
    \begin{subfigure}[b]{0.33\textwidth}
        \centering
        \incfig{block_size_fluidanimate}
        \caption{Fluidanimate.}\label{fig:block_size_fluidanimate}
    \end{subfigure}
    \hfill
    \caption{The graphs show the number of executed cycles when varying the block size. There are three different settings for the block size, 16 bytes, 32 bytes and 64 bytes. The associativity is 2 and the cache size is 4096 bytes.}\label{fig:block_size}
\end{figure}

Given the shifts of the cache size, associativity and the block size it is hard to
identify which of the protocols is the overall winner. It depends on the benchmark,
and the difference is not that great after all if we consider the absolute values
for each parameter in each subplot. They have a similar performance.

Looking at Figure~\ref{fig:inv_bus}, we see a big difference between the two protocols.
The number of invalidations are far more for the MESI protocol, than the number of bus
updates for the Dragon protocol for all benchmarks.



\begin{figure}[H]
    \centering
    \begin{subfigure}[b]{0.5\textwidth}
        \centering
        \incfig{invalidations}
        \caption{Invalidations or bus updates.}\label{fig:invalidations}
    \end{subfigure}%
    \hfill
    \begin{subfigure}[b]{0.5\textwidth}
        \centering
        \incfig{traffic}
        \caption{Bus traffic.}\label{fig:bus_traffic}
    \end{subfigure}%
    \hfill
    \caption{The graphs show the number of invalidations or bus updates and the amount of bus traffic}\label{fig:inv_bus}
\end{figure}

\begin{figure}[H]
    \centering
    \begin{subfigure}[b]{0.5\textwidth}
        \centering
        \incfig{total_private_accesses}
        \caption{Total of private memory accesses.}\label{fig:total_private_accesses}
    \end{subfigure}%
    \hfill
    \begin{subfigure}[b]{0.5\textwidth}
        \centering
        \incfig{total_shared_accesses}
        \caption{Total of shared memory accesses.}\label{fig:total_shared_accesses}
    \end{subfigure}%
    \hfill
    \caption{The graphs show the number of private and shared memory accesses.}\label{fig:accesses}
\end{figure}




\subsection{Advanced Task}\label{results:advanced}

The analysis in this section is again based on the default configuration described in
the project description and the same tests will be conducted as in the previous
section. The difference is that the Dragon protocol is replaced with the
advanced version of the MESI protocol. The difference is calculated using
\[
    P^\text{MESI}_i - P^\text{MESI (advanced)}_i
\]
where $ P^\text{<protocol>}_i $ is the value for each protocol, for each
parameter value $i$ in each plot. A negative difference means that the advanced
version of the MESI protocol required more cycles and a positive difference means
that MESI was slower than the advanced version.

We see the same overall trend when shifting the cache size, associativity and block size
as in the previous section, see
Figures~\ref{fig:cache_size_advanced},~\ref{fig:associativity_advanced} and~\ref{fig:block_size_advanced}. There are more
bars in favor of the advanced version of the MESI protocol.


\begin{figure}[H]
    \centering
    \begin{subfigure}[b]{0.33\textwidth}
        \centering
        \incfig{cache_size_blackscholes_advanced}
        \caption{Blackscholes.}\label{fig:cache_size_blackscholes_advanced}
    \end{subfigure}%
    \hfill
    \begin{subfigure}[b]{0.33\textwidth}
        \centering
        Cache size\par\medskip
        \incfig{cache_size_bodytrack_advanced}
        \caption{Bodytrack.}\label{fig:cache_size_bodytrack_advanced}
    \end{subfigure}%
    \hfill
    \begin{subfigure}[b]{0.33\textwidth}
        \centering
        \incfig{cache_size_fluidanimate_advanced}
        \caption{Fluidanimate.}\label{fig:cache_size_fluidanimate_advanced}
    \end{subfigure}
    \hfill
    \caption{The graphs show the number of executed cycles when varying the size of the cache. There are three different settings for the cache size, 1024 bytes, 4096 bytes and 8192 bytes. The associativity is 2 and the block size is 32 bytes.}\label{fig:cache_size_advanced}
\end{figure}

\begin{figure}[H]
    \centering
    \begin{subfigure}[b]{0.33\textwidth}
        \centering
        \incfig{associativity_blackscholes_advanced}
        \caption{Blackscholes.}\label{fig:associativity_blackscholes_advanced}
    \end{subfigure}%
    \hfill
    \begin{subfigure}[b]{0.33\textwidth}
        \centering
        Associativity\par\medskip
        \incfig{associativity_bodytrack_advanced}
        \caption{Bodytrack.}\label{fig:associativity_bodytrack_advanced}
    \end{subfigure}%
    \hfill
    \begin{subfigure}[b]{0.33\textwidth}
        \centering
        \incfig{associativity_fluidanimate_advanced}
        \caption{Fluidanimate.}\label{fig:associativity_fluidanimate_advanced}
    \end{subfigure}
    \hfill
    \caption{The graphs show the number of executed cycles when varying the associativity. There are three different settings for the associativity, 1 (direct mapped), 2 (2-set-associative) and 128 (fully associative). The block size is 32 bytes and the cache size is 4096 bytes.}\label{fig:associativity_advanced}
\end{figure}

\begin{figure}[H]
    \centering
    \begin{subfigure}[b]{0.33\textwidth}
        \centering
        \incfig{block_size_blackscholes_advanced}
        \caption{Blackscholes.}\label{fig:block_size_blackscholes_advanced}
    \end{subfigure}%
    \hfill
    \begin{subfigure}[b]{0.33\textwidth}
        \centering
        Block size\par\medskip
        \incfig{block_size_bodytrack_advanced}
        \caption{Bodytrack.}\label{fig:block_size_bodytrack_advanced}
    \end{subfigure}%
    \hfill
    \begin{subfigure}[b]{0.33\textwidth}
        \centering
        \incfig{block_size_fluidanimate_advanced}
        \caption{Fluidanimate.}\label{fig:block_size_fluidanimate_advanced}
    \end{subfigure}
    \hfill
    \caption{The graphs show the number of executed cycles when varying the block size. There are three different settings for the block size, 16 bytes, 32 bytes and 64 bytes. The associativity is 2 and the cache size is 4096 bytes.}\label{fig:block_size_advanced}
\end{figure}

\begin{figure}[H]
    \centering
    \begin{subfigure}[b]{0.5\textwidth}
        \centering
        \incfig{invalidations_advanced}
        \caption{Invalidations or bus updates.}\label{fig:invalidations_advanced}
    \end{subfigure}%
    \hfill
    \begin{subfigure}[b]{0.5\textwidth}
        \centering
        \incfig{traffic_advanced}
        \caption{Bus traffic.}\label{fig:bus_traffic_advanced}
    \end{subfigure}%
    \hfill
    \caption{The graphs show the number of invalidations or bus updates and the amount of bus traffic}\label{fig:inv_bus_advanced}
\end{figure}

\begin{figure}[H]
    \centering
    \begin{subfigure}[b]{0.5\textwidth}
        \centering
        \incfig{total_private_accesses_advanced}
        \caption{Total of private memory accesses.}\label{fig:total_private_accesses_advanced}
    \end{subfigure}%
    \hfill
    \begin{subfigure}[b]{0.5\textwidth}
        \centering
        \incfig{total_shared_accesses_advanced}
        \caption{Total of shared memory accesses.}\label{fig:total_shared_accesses_advanced}
    \end{subfigure}%
    \hfill
    \caption{The graphs show the number of private and shared memory accesses.}\label{fig:accesses_advanced}
\end{figure}
