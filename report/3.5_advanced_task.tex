\section{Advanced Task}\label{sec:advanced_task}
While implementing the two previously described protocols, we quickly noticed that both protocols
leave potential for optimizations. Most of their inefficiencies could be removed by better
distributing information between the caches and their controllers. This however introduces more bus
traffic or more complexity into the processor design. Researchers~\cite{read_broadcast_prop1,
    read_broadcast_prop2, read_broadcast_prop3} therefore introduced a new
optimization that does not introduce additional bus traffic and is of minor complexity. They
proposed a technique called \emph{Read-Broadcast}~\cite{read_broadcast_analysis} for snooping and
invalidation based cache coherency systems.

Suppose we have a system with multiple cores that each hold the same block in their caches. We also
assume that at some point one of the cores initiates a write to this block and therefore sends an
invalidation signal to all the other caches. If one of the other cores now were to issue another
read to the same block, this read would result in a cache miss because of the previously received
invalidation of the cache block. This read miss occurs for every one of the reading cores that got
invalidated and all of them would need to re-read the block's value from memory.

In a cache coherency system with the \emph{Read-Broadcast} optimization, all caches snoop on the bus
line to detect reads of cache blocks they currently hold. If their stored version is marked as
invalid, they replace it with the block that is currently sent over the bus.

We implemented this optimization for our simulator and evaluated its performance improvements in
Section~\ref{results:advanced}.
